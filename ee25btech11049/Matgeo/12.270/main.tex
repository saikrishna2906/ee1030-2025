\let\negmedspace\undefined
\let\negthickspace\undefined
\documentclass[journal]{IEEEtran}
\usepackage[a5paper, margin=10mm, onecolumn]{geometry}
%\usepackage{lmodern} % Ensure lmodern is loaded for pdflatex
\usepackage{tfrupee} % Include tfrupee package

\setlength{\headheight}{1cm} % Set the height of the header box
\setlength{\headsep}{0mm}     % Set the distance between the header box and the top of the text

\usepackage{gvv-book}
\usepackage{gvv}
\usepackage{cite}
\usepackage{amsmath,amssymb,amsfonts,amsthm}
\usepackage{algorithmic}
\usepackage{graphicx}
\usepackage{textcomp}
\usepackage{xcolor}
\usepackage{txfonts}
\usepackage{listings}
\usepackage{enumitem}
\usepackage{mathtools}
\usepackage{gensymb}
\usepackage{comment}
\usepackage[breaklinks=true]{hyperref}
\usepackage{tkz-euclide} 
\usepackage{listings}
% \usepackage{gvv}                                        
\def\inputGnumericTable{}                                 
\usepackage[latin1]{inputenc}                                
\usepackage{color}                                            
\usepackage{array}                                            
\usepackage{longtable}                                       
\usepackage{calc}                                             
\usepackage{multirow}                                         
\usepackage{hhline}                                           
\usepackage{ifthen}                                           
\usepackage{lscape}
\begin{document}

\bibliographystyle{IEEEtran}
\vspace{3cm}

\title{12.270}
\author{EE25BTECH11049 - Sai Krishna Bakki}
\maketitle
\vspace{-3em}
\textbf{Question:}\\
If
\begin{align*}
    \vec{A}=\myvec{2&4\\1&3},\vec{B}=\myvec{4&6\\5&9},
\end{align*}
$\brak{\vec{A}\vec{B}}^T$ is equal to\\
\solution\\
Given
\begin{align}
    \vec{A}=\myvec{2&4\\1&3},\vec{B}=\myvec{4&6\\5&9}\\
    \vec{A}^T=\myvec{2&1\\4&3},\vec{B}^T=\myvec{4&5\\6&9}
\end{align}
$\brak{\vec{A}\vec{B}}^T$ can also be written as $\vec{B}^T\vec{A}^T$
\begin{align}
    \brak{\vec{A}\vec{B}}^T=\vec{B}^T\vec{A}^T\\
    \implies \myvec{4&5\\6&9}\myvec{2&1\\4&3}\\
    \implies\myvec{8+20&4+15\\12+36&6+27}\\
    \implies \myvec{28&19\\48&33}
\end{align}
$\therefore$ $ \brak{\vec{A}\vec{B}}^T$ is equal to  $\myvec{28&19\\48&33}$.
\end{document}