\documentclass{beamer}
\usepackage[utf8]{inputenc}

\usetheme{Madrid}
\usecolortheme{default}
\usepackage{amsmath,amssymb,amsfonts,amsthm}
\usepackage{txfonts}
\usepackage{tkz-euclide}
\usepackage{listings}
\usepackage{adjustbox}
\usepackage{array}
\usepackage{tabularx}
\usepackage{gvv}
\usepackage{gvv}
\usepackage{lmodern}
\usepackage{circuitikz}
\usepackage{tikz}
\usepackage{graphicx}

\setbeamertemplate{page number in head/foot}[totalframenumber]

\usepackage{tcolorbox}
\tcbuselibrary{minted,breakable,xparse,skins}



\definecolor{bg}{gray}{0.95}
\DeclareTCBListing{mintedbox}{O{}m!O{}}{%
  breakable=true,
  listing engine=minted,
  listing only,
  minted language=#2,
  minted style=default,
  minted options={%
    linenos,
    gobble=0,
    breaklines=true,
    breakafter=,,
    fontsize=\small,
    numbersep=8pt,
    #1},
  boxsep=0pt,
  left skip=0pt,
  right skip=0pt,
  left=25pt,
  right=0pt,
  top=3pt,
  bottom=3pt,
  arc=5pt,
  leftrule=0pt,
  rightrule=0pt,
  bottomrule=2pt,
  toprule=2pt,
  colback=bg,
  colframe=orange!70,
  enhanced,
  overlay={%
    \begin{tcbclipinterior}
    \fill[orange!20!white] (frame.south west) rectangle ([xshift=20pt]frame.north west);
    \end{tcbclipinterior}},
  #3,
}
\lstset{
    language=C,
    basicstyle=\ttfamily\small,
    keywordstyle=\color{blue},
    stringstyle=\color{orange},
    commentstyle=\color{green!60!black},
    numbers=left,
    numberstyle=\tiny\color{gray},
    breaklines=true,
    showstringspaces=false,
}
%------------------------------------------------------------
%This block of code defines the information to appear in the
%Title page
\title %optional
{12.62}
\date{}
%\subtitle{A short story}

\author % (optional)
{Sai Krishna Bakki - EE25BTECH11049}

\begin{document}

\frame{\titlepage}
\begin{frame}{Question}
The eigenvalues of the matrix 
\begin{align*}
    \myvec{2&3&0\\3&2&0\\0&0&1}
\end{align*}
are
\end{frame}
\begin{frame}{Theoretical Solution}
    Given\\
\begin{align}
    \vec{A}=\myvec{2&3&0\\3&2&0\\0&0&1}
\end{align}
To find eigenvalues of the matrix $\vec{A}$
\begin{align}
    \vec{A}\vec{x}=\lambda\vec{x}\\
    \brak{\vec{A}-\lambda\vec{I}}\vec{x}=0\\
    \mydet{\vec{A}-\lambda\vec{I}}=0
    \end{align}
\end{frame}
\begin{frame}{Theoretical Solution}
\begin{align}
    \mydet{2-\lambda&3&0\\3&2-\lambda&0\\0&0&1-\lambda}=0\\
    (2-\lambda)\brak{(2-\lambda)(1-\lambda)-0}-3(3)(1-\lambda)=0\\
    (1-\lambda)\brak{(2-\lambda)^2-9}=0\\
    \lambda=1,-1,5
\end{align}
$\therefore$ The eigenvalues of the matrix are 1,-1 and 5.
\end{frame}
\begin{frame}[fragile]
\frametitle{C Code}
\begin{lstlisting}
#include<math.h>
    void find_2x2_eigenvalues(double a, double b, double c, double d, double* eig1, double* eig2) {
    // For the equation x^2 + Bx + C = 0, the solutions are (-B +/- sqrt(B^2 - 4C)) / 2.
    // Here, B = -(a+d) and C = (ad-bc).
    double trace = a + d;
    double determinant = a * d - b * c;

    // Calculate the discriminant: sqrt(trace^2 - 4*determinant)
    double discriminant_sqrt = sqrt(trace * trace - 4 * determinant);

    // Calculate the two eigenvalues using the formula
    *eig1 = (trace + discriminant_sqrt) / 2.0;
    *eig2 = (trace - discriminant_sqrt) / 2.0;
}
\end{lstlisting}    
\end{frame}
\begin{frame}[fragile]
\frametitle{Python Code Through Shared Output }
\begin{lstlisting}
import ctypes
import numpy as np
import os
import platform

# --- Step 1: Compile the C code into a shared library ---
# This script will attempt to compile the C code automatically.
# The C source file is expected to be 'eigenv.c'.

c_file_name = 'eigenv.c'

# Determine the correct file extension for the shared library based on the OS
if platform.system() == "Windows":
    lib_name = 'eigen_lib.dll'
    compile_command = f"gcc -shared -o {lib_name} -fPIC {c_file_name}"
elif platform.system() == "Darwin": # macOS
    lib_name = 'eigen_lib.dylib'
    \end{lstlisting}    
\end{frame}
\begin{frame}[fragile]
\frametitle{Python Code Through Shared Output }
\begin{lstlisting}
    compile_command = f"gcc -shared -o {lib_name} -fPIC {c_file_name}"
else: # Linux
    lib_name = 'eigenv.so'
    compile_command = f"gcc -shared -o {lib_name} -fPIC {c_file_name}"

# Compile the C code if the library file doesn't exist
if not os.path.exists(lib_name):
    print(f"Shared library '{lib_name}' not found. Attempting to compile '{c_file_name}'...")
    exit_code = os.system(compile_command)
    if exit_code != 0:
        print(f"\nError: Compilation failed. Please ensure GCC is installed and in your system's PATH.")
        print(f"Manual compile command: {compile_command}")
        exit()
    print("Compilation successful.")
\end{lstlisting}    
\end{frame}
\begin{frame}[fragile]
\frametitle{Python Code Through Shared Output }
\begin{lstlisting}
# --- Step 2: Load the shared library using ctypes ---
try:
    # Use the absolute path to ensure the library is found
    eigen_lib = ctypes.CDLL(os.path.abspath(lib_name))
except OSError as e:
    print(f"Error loading shared library: {e}")
    exit()

# --- Step 3: Define the function signature (argument and return types) ---
# The C function is:
# void find_2x2_eigenvalues(double a, double b, double c, double d, double* eig1, double* eig2)
find_2x2_eigenvalues_c = eigen_lib.find_2x2_eigenvalues
find_2x2_eigenvalues_c.argtypes = [
    ctypes.c_double, ctypes.c_double,
    ctypes.c_double, ctypes.c_double,
    ctypes.POINTER(ctypes.c_double),
    ctypes.POINTER(ctypes.c_double)]
    \end{lstlisting}    
\end{frame}
\begin{frame}[fragile]
\frametitle{Python Code Through Shared Output }
\begin{lstlisting}
find_2x2_eigenvalues_c.restype = None  # Corresponds to a 'void' return type in C
# --- Step 4: Prepare data and call the C function ---

# The full 3x3 matrix is block-diagonal, so we can analyze it in parts.
# [[2, 3, 0],
#  [3, 2, 0],
#  [0, 0, 1]]
# One eigenvalue is 1. The other two come from the top-left 2x2 sub-matrix.
sub_matrix = np.array([[2, 3], [3, 2]])
a, b = sub_matrix[0]
c, d = sub_matrix[1]

# Create C-compatible double variables to hold the results from the C function
eig1_c = ctypes.c_double()
eig2_c = ctypes.c_double()
\end{lstlisting}    
\end{frame}
\begin{frame}[fragile]
\frametitle{Python Code Through Shared Output }
\begin{lstlisting}
print(f"Calling C function to find eigenvalues of the sub-matrix:\n{sub_matrix}\n")
# Call the C function, passing pointers to the result variables
find_2x2_eigenvalues_c(a, b, c, d, ctypes.byref(eig1_c), ctypes.byref(eig2_c))

# --- Step 5: Retrieve the results and combine them ---
eigenvalues_from_c = [eig1_c.value, eig2_c.value]
third_eigenvalue = 1.0
all_eigenvalues = eigenvalues_from_c + [third_eigenvalue]

# Sort for consistent output
all_eigenvalues.sort(reverse=True)
print(f"Eigenvalues from C function: {eigenvalues_from_c}")
print(f"Third eigenvalue from observation: {third_eigenvalue}")
print(f"\nFinal eigenvalues for the 3x3 matrix are: {all_eigenvalues}")
\end{lstlisting}    
\end{frame}
\begin{frame}[fragile]
\frametitle{Python Code}
\begin{lstlisting}
import numpy as np

A = np.array([[2, 3, 0],
              [3, 2, 0],
              [0, 0, 1]])

# Use numpy's linear algebra module (linalg) to find the eigenvalues.
# The function eigvals() returns the eigenvalues of a square matrix.
eigenvalues = np.linalg.eigvals(A)

# Print the original matrix and the calculated eigenvalues.
print("Matrix:")
print(A)
print("\nEigenvalues:")
print(eigenvalues)
\end{lstlisting}
\end{frame}
\end{document}