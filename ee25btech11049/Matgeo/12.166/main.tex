\let\negmedspace\undefined
\let\negthickspace\undefined
\documentclass[journal]{IEEEtran}
\usepackage[a5paper, margin=10mm, onecolumn]{geometry}
%\usepackage{lmodern} % Ensure lmodern is loaded for pdflatex
\usepackage{tfrupee} % Include tfrupee package

\setlength{\headheight}{1cm} % Set the height of the header box
\setlength{\headsep}{0mm}     % Set the distance between the header box and the top of the text

\usepackage{gvv-book}
\usepackage{gvv}
\usepackage{cite}
\usepackage{amsmath,amssymb,amsfonts,amsthm}
\usepackage{algorithmic}
\usepackage{graphicx}
\usepackage{textcomp}
\usepackage{xcolor}
\usepackage{txfonts}
\usepackage{listings}
\usepackage{enumitem}
\usepackage{mathtools}
\usepackage{gensymb}
\usepackage{comment}
\usepackage[breaklinks=true]{hyperref}
\usepackage{tkz-euclide} 
\usepackage{listings}
% \usepackage{gvv}                                        
\def\inputGnumericTable{}                                 
\usepackage[latin1]{inputenc}                                
\usepackage{color}                                            
\usepackage{array}                                            
\usepackage{longtable}                                       
\usepackage{calc}                                             
\usepackage{multirow}                                         
\usepackage{hhline}                                           
\usepackage{ifthen}                                           
\usepackage{lscape}
\begin{document}

\bibliographystyle{IEEEtran}
\vspace{3cm}

\title{12.166}
\author{EE25BTECH11049 - Sai Krishna Bakki}
\maketitle
\vspace{-3em}
\textbf{Question:}\\
Let $\vec{R}$ be an $n \times n$ nonsingular matrix. Let $\vec{P}$ and $\vec{Q}$ be two $n \times n$ matrices such that $\vec{Q} = \vec{R}^{-1}\vec{P}\vec{R}$. If $\vec{x}$ is an eigenvector of $\vec{P}$ corresponding to a nonzero eigenvalue $\lambda$ of $\vec{P}$, then
\begin{enumerate}[label=\alph*)]
    \item $\vec{R}\vec{x}$ is an eigenvector of $\vec{Q}$ corresponding to eigenvalue $\lambda$ of $\vec{Q}$
    \item $\vec{R}\vec{x}$ is an eigenvector of $\vec{Q}$ corresponding to eigenvalue $\frac{1}{\lambda}$ of $\vec{Q}$
    \item $\vec{R}^{-1}\vec{x}$ is an eigenvector of $\vec{Q}$ corresponding to eigenvalue $\lambda$ of $\vec{Q}$
    \item $\vec{R}^{-1}\vec{x}$ is an eigenvector of $\vec{Q}$ corresponding to eigenvalue $\frac{1}{\lambda}$ of $\vec{Q}$
\end{enumerate}


\solution\\
 Start with the eigenvector equation for P:
    \begin{align}
    \vec{P}\vec{x} = \lambda \vec{x}
\end{align}
Pre-multiply both sides by $\vec{R}^{-1}$:
    \begin{align} \vec{R}^{-1}(\vec{P}\vec{x}) = \vec{R}^{-1}(\lambda \vec{x}) 
    \end{align}

    % \item Since $\lambda$ is a scalar, it can be moved outside the matrix multiplication:
    % $$ \vec{R}^{-1}\vec{P}\vec{x} = \lambda(\vec{R}^{-1}\vec{x}) $$

Now, we need to relate this to \textbf{Q}. We know that $\vec{Q} = \vec{R}^{-1}\vec{P}\vec{R}$. Let's introduce the identity matrix $\vec{I} = \vec{R}\vec{R}^{-1}$ into our equation.
    \begin{align} \vec{R}^{-1}\vec{P}(\vec{R}\vec{R}^{-1})\vec{x} = \lambda(\vec{R}^{-1}\vec{x}) \\ (\vec{R}^{-1}\vec{P}\vec{R})(\vec{R}^{-1}\vec{x}) = \lambda(\vec{R}^{-1}\vec{x}) 
\end{align}
Substitute $\vec{Q} = \vec{R}^{-1}\vec{P}\vec{R}$ into the equation:
    \begin{align} \vec{Q}(\vec{R}^{-1}\vec{x}) = \lambda(\vec{R}^{-1}\vec{x}) \end{align}

$\therefore$ $\vec{R}^{-1}\vec{x}$ is an eigenvector of $\vec{Q}$ corresponding to the eigenvalue $\lambda$ of $\vec{Q}$.
\end{document}
