\documentclass{beamer}
\usepackage[utf8]{inputenc}

\usetheme{Madrid}
\usecolortheme{default}
\usepackage{extarrows}
\usepackage{amsmath}
\usepackage{extarrows}
\usepackage{amssymb,amsfonts,amsthm}
\usepackage{txfonts}
\usepackage{tkz-euclide}
\usepackage{listings}
\usepackage{adjustbox}
\usepackage{array}
\usepackage{tabularx}
\usepackage{gvv}
\usepackage{lmodern}
\usepackage{circuitikz}
\usepackage{tikz}
\usepackage{graphicx}
\usepackage{amsmath} 

\setbeamertemplate{page number in head/foot}[totalframenumber]

\usepackage{tcolorbox}
\tcbuselibrary{minted,breakable,xparse,skins}

\definecolor{bg}{gray}{0.95}
\DeclareTCBListing{mintedbox}{O{}m!O{}}{%
  breakable=true,
  listing engine=minted,
  listing only,
  minted language=#2,
  minted style=default,
  minted options={%
    linenos,
    gobble=0,
    breaklines=true,
    breakafter=,,
    fontsize=\small,
    numbersep=8pt,
    #1},
  boxsep=0pt,
  left skip=0pt,
  right skip=0pt,
  left=25pt,
  right=0pt,
  top=3pt,
  bottom=3pt,
  arc=5pt,
  leftrule=0pt,
  rightrule=0pt,
  bottomrule=2pt,
  toprule=2pt,
  colback=bg,
  colframe=orange!70,
  enhanced,
  overlay={%
    \begin{tcbclipinterior}
    \fill[orange!20!white] (frame.south west) rectangle ([xshift=20pt]frame.north west);
    \end{tcbclipinterior}},
  #3,
}
\lstset{
    language=C,
    basicstyle=\ttfamily\small,
    keywordstyle=\color{blue},
    stringstyle=\color{orange},
    commentstyle=\color{green!60!black},
    numbers=left,
    numberstyle=\tiny\color{gray},
    breaklines=true,
    showstringspaces=false,
}
\title %optional
{12.166}
\author 
{EE25BTECH11049-Sai Krishna Bakki}

\begin{document}

\frame{\titlepage}
\begin{frame}{Question}
Let $\vec{R}$ be an $n \times n$ nonsingular matrix. Let $\vec{P}$ and $\vec{Q}$ be two $n \times n$ matrices such that $\vec{Q} = \vec{R}^{-1}\vec{P}\vec{R}$. If $\vec{x}$ is an eigenvector of $\vec{P}$ corresponding to a nonzero eigenvalue $\lambda$ of $\vec{P}$, then
\begin{enumerate}[label=\alph*)]
    \item $\vec{R}\vec{x}$ is an eigenvector of $\vec{Q}$ corresponding to eigenvalue $\lambda$ of $\vec{Q}$
    \item $\vec{R}\vec{x}$ is an eigenvector of $\vec{Q}$ corresponding to eigenvalue $\frac{1}{\lambda}$ of $\vec{Q}$
    \item $\vec{R}^{-1}\vec{x}$ is an eigenvector of $\vec{Q}$ corresponding to eigenvalue $\lambda$ of $\vec{Q}$
    \item $\vec{R}^{-1}\vec{x}$ is an eigenvector of $\vec{Q}$ corresponding to eigenvalue $\frac{1}{\lambda}$ of $\vec{Q}$
\end{enumerate}
\end{frame}
\begin{frame}{Theoretical Solution}
    \begin{enumerate*}
    \item Start with the eigenvector equation for P:
    \begin{align}
    \vec{P}\vec{x} = \lambda \vec{x}
\end{align}
    \item Pre-multiply both sides by $\vec{R}^{-1}$:
    \begin{align} \vec{R}^{-1}(\vec{P}\vec{x}) = \vec{R}^{-1}(\lambda \vec{x}) 
    \end{align}
    \item Now, we need to relate this to \textbf{Q}. We know that $\vec{Q} = \vec{R}^{-1}\vec{P}\vec{R}$. Let's introduce the identity matrix $\vec{I} = \vec{R}\vec{R}^{-1}$ into our equation.
    \begin{align} \vec{R}^{-1}\vec{P}(\vec{R}\vec{R}^{-1})\vec{x} = \lambda(\vec{R}^{-1}\vec{x}) \\ (\vec{R}^{-1}\vec{P}\vec{R})(\vec{R}^{-1}\vec{x}) = \lambda(\vec{R}^{-1}\vec{x}) 
\end{align}
    \item Substitute $\vec{Q} = \vec{R}^{-1}\vec{P}\vec{R}$ into the equation:
    \begin{align} \vec{Q}(\vec{R}^{-1}\vec{x}) = \lambda(\vec{R}^{-1}\vec{x}) \end{align}
\end{enumerate*}

$\therefore$ $\vec{R}^{-1}\vec{x}$ is an eigenvector of $\vec{Q}$ corresponding to the eigenvalue $\lambda$ of $\vec{Q}$.
\end{frame}
\end{document}