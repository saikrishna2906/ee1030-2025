\documentclass{beamer}
\usepackage[utf8]{inputenc}

\usetheme{Madrid}
\usecolortheme{default}
\usepackage{amsmath,amssymb,amsfonts,amsthm}
\usepackage{txfonts}
\usepackage{tkz-euclide}
\usepackage{listings}
\usepackage{adjustbox}
\usepackage{array}
\usepackage{tabularx}
\usepackage{gvv}
\usepackage{gvv}
\usepackage{lmodern}
\usepackage{circuitikz}
\usepackage{tikz}
\usepackage{graphicx}

\setbeamertemplate{page number in head/foot}[totalframenumber]

\usepackage{tcolorbox}
\tcbuselibrary{minted,breakable,xparse,skins}



\definecolor{bg}{gray}{0.95}
\DeclareTCBListing{mintedbox}{O{}m!O{}}{%
  breakable=true,
  listing engine=minted,
  listing only,
  minted language=#2,
  minted style=default,
  minted options={%
    linenos,
    gobble=0,
    breaklines=true,
    breakafter=,,
    fontsize=\small,
    numbersep=8pt,
    #1},
  boxsep=0pt,
  left skip=0pt,
  right skip=0pt,
  left=25pt,
  right=0pt,
  top=3pt,
  bottom=3pt,
  arc=5pt,
  leftrule=0pt,
  rightrule=0pt,
  bottomrule=2pt,
  toprule=2pt,
  colback=bg,
  colframe=orange!70,
  enhanced,
  overlay={%
    \begin{tcbclipinterior}
    \fill[orange!20!white] (frame.south west) rectangle ([xshift=20pt]frame.north west);
    \end{tcbclipinterior}},
  #3,
}
\lstset{
    language=C,
    basicstyle=\ttfamily\small,
    keywordstyle=\color{blue},
    stringstyle=\color{orange},
    commentstyle=\color{green!60!black},
    numbers=left,
    numberstyle=\tiny\color{gray},
    breaklines=true,
    showstringspaces=false,
}
%------------------------------------------------------------
%This block of code defines the information to appear in the
%Title page
\title %optional
{12.894}
\date{}
%\subtitle{A short story}

\author % (optional)
{Sai Krishna Bakki - EE25BTECH11049}

\begin{document}

\frame{\titlepage}
\begin{frame}{Question}
The eigenvalues of the matrix 
\begin{align*}
\myvec{0&-1\\1&0}
\end{align*}
are
\end{frame}
\begin{frame}{Theoretical Solution}
Given\\
\begin{align}
    \vec{A}=\myvec{0&-1\\1&0}
\end{align}
To find eigenvalues of the matrix $\vec{A}$
\begin{align}
    \vec{A}\vec{x}=\lambda\vec{x}\\
    \brak{\vec{A}-\lambda\vec{I}}\vec{x}=0\\
    \mydet{\vec{A}-\lambda\vec{I}}=0\\
    \mydet{0-\lambda&-1\\1&0-\lambda}=0\\
    (-\lambda)(-\lambda)-(-1)(1)\\
    \lambda^2=-1\\
    \lambda=-\sqrt{-1},\sqrt{-1}
    \end{align}
    $\therefore$ The eigenvalues of the matrix are $\lambda=-\sqrt{-1},\sqrt{-1}$.
\end{frame}
\begin{frame}[fragile]
\frametitle{C Code}
\begin{lstlisting}
    #include <math.h>

void find_2x2_eigenvalues(double a, double b, double c, double d,
                                double* eig1_real, double* eig1_imag,
                                double* eig2_real, double* eig2_imag) {
    double trace = a + d;
    double determinant = a * d - b * c;
    double discriminant = trace * trace - 4 * determinant;

    if (discriminant >= 0) {
        // Real eigenvalues
        double sqrt_discriminant = sqrt(discriminant);
        *eig1_real = (trace + sqrt_discriminant) / 2.0;
        *eig1_imag = 0.0;
        *eig2_real = (trace - sqrt_discriminant) / 2.0;
        *eig2_imag = 0.0;
\end{lstlisting}
\end{frame}
\begin{frame}[fragile]
\frametitle{C Code}
\begin{lstlisting}
    } else {
        // Complex conjugate eigenvalues
        double sqrt_abs_discriminant = sqrt(-discriminant);
        *eig1_real = trace / 2.0;
        *eig1_imag = sqrt_abs_discriminant / 2.0;
        *eig2_real = trace / 2.0;
        *eig2_imag = -sqrt_abs_discriminant / 2.0;
    }
}
\end{lstlisting}
\end{frame}
\begin{frame}[fragile]
\frametitle{Python Code Through Shared Output}
\begin{lstlisting}
import ctypes
import numpy as np
import os
import platform

# --- Step 1: Compile the C code into a shared library ---
c_file_name = 'eigenv.c'

# Determine the correct file extension for the shared library and compile command
if platform.system() == "Windows":
    lib_name = 'eigen_lib.dll'
    compile_command = f"gcc -shared -o {lib_name} -fPIC {c_file_name}"
elif platform.system() == "Darwin": # macOS
    lib_name = 'eigen_lib.dylib'
    compile_command = f"gcc -shared -o {lib_name} -fPIC {c_file_name}"
else: # Linux
\end{lstlisting}
\end{frame}
\begin{frame}[fragile]
\frametitle{Python Code Through Shared Output}
\begin{lstlisting}
    lib_name = 'eigenv.so'
    # Add -lm to link the math library on Linux/macOS
    compile_command = f"gcc -shared -o {lib_name} -fPIC {c_file_name} -lm"

# Compile the C code if the library file doesn't exist
if not os.path.exists(lib_name):
    print(f"Shared library not found. Compiling '{c_file_name}'...")
    if os.system(compile_command) != 0:
        print(f"\nError: Compilation failed. Please ensure GCC is installed.")
        exit()
    print("Compilation successful.")

# --- Step 2: Load the shared library ---
try:
    eigen_lib = ctypes.CDLL(os.path.abspath(lib_name))
except OSError as e:
\end{lstlisting}
\end{frame}
\begin{frame}[fragile]
\frametitle{Python Code Through Shared Output}
\begin{lstlisting}
    print(f"Error loading shared library: {e}")
    exit()

# --- Step 3: Define the function signature ---
# The modified C function signature is:
# void find_2x2_eigenvalues(double, double, double, double, double*, double*, double*, double*)
find_2x2_c = eigen_lib.find_2x2_eigenvalues
find_2x2_c.argtypes = [
    ctypes.c_double, ctypes.c_double, ctypes.c_double, ctypes.c_double,
    ctypes.POINTER(ctypes.c_double), ctypes.POINTER(ctypes.c_double),
    ctypes.POINTER(ctypes.c_double), ctypes.POINTER(ctypes.c_double)
]
find_2x2_c.restype = None
# --- Step 4: Prepare data and call the C function ---
# The matrix from the image is:
\end{lstlisting}
\end{frame}
\begin{frame}[fragile]
\frametitle{Python Code Through Shared Output}
\begin{lstlisting}
# [[0, -1],
#  [1,  0]]
matrix = np.array([[0, -1], [1, 0]])
a, b = float(matrix[0, 0]), float(matrix[0, 1])
c, d = float(matrix[1, 0]), float(matrix[1, 1])

# Create C-compatible double variables to hold the real and imaginary parts
eig1_real, eig1_imag = ctypes.c_double(), ctypes.c_double()
eig2_real, eig2_imag = ctypes.c_double(), ctypes.c_double()

print(f"Calling C function to find eigenvalues of the matrix:\n{matrix}\n")
\end{lstlisting}
\end{frame}
\begin{frame}[fragile]
\frametitle{Python Code Through Shared Output}
\begin{lstlisting}
# Call the C function, passing pointers to the result variables
find_2x2_c(a, b, c, d,
           ctypes.byref(eig1_real), ctypes.byref(eig1_imag),
           ctypes.byref(eig2_real), ctypes.byref(eig2_imag))

# --- Step 5: Retrieve the results and combine them into complex numbers ---
eigenvalue1 = complex(eig1_real.value, eig1_imag.value)
eigenvalue2 = complex(eig2_real.value, eig2_imag.value)

print(f"Eigenvalues from C function are: {eigenvalue1} and {eigenvalue2}")
\end{lstlisting}
\end{frame}
\begin{frame}[fragile]
\frametitle{Python Code}
\begin{lstlisting}
import numpy as np
A = np.array([[0, -1],
              [1,  0]])

# Use numpy's linear algebra module to find the eigenvalues.
eigenvalues = np.linalg.eigvals(A)

# Print the original matrix.
print("Matrix:")
print(A)
formatted_vals = [f"{int(val.imag)}j" for val in eigenvalues]

print("\nEigenvalues:")
print(f"The eigenvalues are {formatted_vals[0]} and {formatted_vals[1]}")
\end{lstlisting}
\end{frame}
\end{document}