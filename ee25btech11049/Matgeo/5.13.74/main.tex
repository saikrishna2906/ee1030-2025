\let\negmedspace\undefined
\let\negthickspace\undefined
\documentclass[journal]{IEEEtran}
\usepackage[a5paper, margin=10mm, onecolumn]{geometry}
%\usepackage{lmodern} % Ensure lmodern is loaded for pdflatex
\usepackage{tfrupee} % Include tfrupee package

\setlength{\headheight}{1cm} % Set the height of the header box
\setlength{\headsep}{0mm}     % Set the distance between the header box and the top of the text

\usepackage{gvv-book}
\usepackage{gvv}
\usepackage{cite}
\usepackage{amsmath,amssymb,amsfonts,amsthm}
\usepackage{algorithmic}
\usepackage{graphicx}
\usepackage{textcomp}
\usepackage{xcolor}
\usepackage{txfonts}
\usepackage{listings}
\usepackage{enumitem}
\usepackage{mathtools}
\usepackage{gensymb}
\usepackage{comment}
\usepackage[breaklinks=true]{hyperref}
\usepackage{tkz-euclide} 
\usepackage{listings}
\usepackage{gvv}                                        
\def\inputGnumericTable{}                                 
\usepackage[latin1]{inputenc}                                
\usepackage{color}                                            
\usepackage{array}                                            
\usepackage{longtable}                                       
\usepackage{calc}                                             
\usepackage{multirow}                                         
\usepackage{hhline}                                           
\usepackage{ifthen}                                           
\usepackage{lscape}
\usepackage{circuitikz}
\tikzstyle{block} = [rectangle, draw, fill=blue!20, 
    text width=4em, text centered, rounded corners, minimum height=3em]
\tikzstyle{sum} = [draw, fill=blue!10, circle, minimum size=1cm, node distance=1.5cm]
\tikzstyle{input} = [coordinate]
\tikzstyle{output} = [coordinate]


\begin{document}

\bibliographystyle{IEEEtran}
\vspace{3cm}

\title{5.13.74}
\author{EE25BTECH11049 - Sai Krishna Bakki}
\maketitle
\vspace{-3em}
% \newpage
% \bigskip
{\let\newpage\relax\maketitle}

\renewcommand{\thefigure}{\theenumi}
\renewcommand{\thetable}{\theenumi}
\setlength{\intextsep}{10pt} % Space between text and floats


\numberwithin{equation}{enumi}
\numberwithin{figure}{enumi}
\renewcommand{\thetable}{\theenumi}

\textbf{Question:}\\
The trace of a square matrix is defined to be the sum of its diagonal entries. If $\vec{A}$ is a 2 x 2 matrix, such that the trace of $\vec{A}$ is 3 and the trace of $\vec{A}^{3}$ is -18, then the value of the determinant of $\vec{A}$ is\\
\textbf{Solution:}\\
Given:
\begin{align}
    \operatorname{tr}(\vec{A})&=3\\
    \operatorname{tr}(\vec{A^3})&=-18\\
    \operatorname{tr}(\vec{I})&=2
\end{align}
Using Cayley-Hamilton Theorem, we know that
\begin{align}
    \mydet{\vec{A}-\lambda\vec{I}}&=0,
    \text{for $\lambda=\vec{A}$}		
\end{align}
For a 2 x 2 matrix $\vec{A}$, the characteristic equation is:
\begin{align}
    \lambda^{2}-\operatorname{tr}(\vec{A})\lambda+\mydet{\vec{A}}&=0
\end{align}
According to the theorem, the matrix $\vec{A}$ itself will satisfy this equation:
\begin{align}
    \vec{A^2}-\operatorname{tr}(\vec{A})\vec{A}+\mydet{\vec{A}}\vec{I}&=0\\
        \vec{A^2}-3\vec{A}+\mydet{\vec{A}}\vec{I}&=0\\
        \vec{A^2}&=3\vec{A}-\mydet{\vec{A}}\vec{I}
\end{align}
To find $\vec{A^3}$, we multiply the equation by $\vec{A}$
\begin{align}
    % \vec{A^3}=\vec{A}\brak{3\vec{A}-\mydet{\vec{A}}\vec{I}}\\
    \vec{A^3}=3\vec{A^2}-\mydet{\vec{A}}\vec{A}
\end{align}
Now, substitute the expression for $\vec{A^2}$ into the equation for $\vec{A^3}$:
\begin{align}
\vec{A^3}=3\brak{3\vec{A}-\mydet{\vec{A}}\vec{I}}-\mydet{\vec{A}}\vec{A}\\
\vec{A^3}=\brak{9-\mydet{\vec{A}}}\vec{A}-3\mydet{\vec{A}}\vec{I}
\end{align}
Let's take the trace of both sides of this equation. Using the linearity properties of the trace 
\begin{align}
    \operatorname{tr}(\vec{X}+\vec{Y})=\operatorname{tr}(\vec{X})+\operatorname{tr}(\vec{Y})\\
    \operatorname{tr}(k\vec{X})=k\operatorname{tr}(\vec{X})\\
    \operatorname{tr}(\vec{A^3})=\operatorname{tr}\brak{\brak{9-\mydet{\vec{A}}}\vec{A}-3\mydet{\vec{A}}\vec{I}}\\
    \operatorname{tr}(\vec{A^3})=\brak{9-\mydet{\vec{A}}}\operatorname{tr}(\vec{A})-3\mydet{\vec{A}}\operatorname{tr}(\vec{I})
\end{align}
Substituting equations (0.1) and (0.2) in above equation, we get
\begin{align}
    -18=\brak{9-\mydet{\vec{A}}}(3)-3\mydet{\vec{A}}(2)\\
    \therefore \mydet{\vec{A}}=5
\end{align}
\end{document}
