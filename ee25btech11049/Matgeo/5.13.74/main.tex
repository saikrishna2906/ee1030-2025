\let\negmedspace\undefined
\let\negthickspace\undefined
\documentclass[journal]{IEEEtran}
\usepackage[a5paper, margin=10mm, onecolumn]{geometry}
%\usepackage{lmodern} % Ensure lmodern is loaded for pdflatex
\usepackage{tfrupee} % Include tfrupee package

\setlength{\headheight}{1cm} % Set the height of the header box
\setlength{\headsep}{0mm}     % Set the distance between the header box and the top of the text

\usepackage{gvv-book}
\usepackage{gvv}
\usepackage{cite}
\usepackage{amsmath,amssymb,amsfonts,amsthm}
\usepackage{algorithmic}
\usepackage{graphicx}
\usepackage{textcomp}
\usepackage{xcolor}
\usepackage{txfonts}
\usepackage{listings}
\usepackage{enumitem}
\usepackage{mathtools}
\usepackage{gensymb}
\usepackage{comment}
\usepackage[breaklinks=true]{hyperref}
\usepackage{tkz-euclide} 
\usepackage{listings}
\usepackage{gvv}                                        
\def\inputGnumericTable{}                                 
\usepackage[latin1]{inputenc}                                
\usepackage{color}                                            
\usepackage{array}                                            
\usepackage{longtable}                                       
\usepackage{calc}                                             
\usepackage{multirow}                                         
\usepackage{hhline}                                           
\usepackage{ifthen}                                           
\usepackage{lscape}
\usepackage{circuitikz}
\tikzstyle{block} = [rectangle, draw, fill=blue!20, 
    text width=4em, text centered, rounded corners, minimum height=3em]
\tikzstyle{sum} = [draw, fill=blue!10, circle, minimum size=1cm, node distance=1.5cm]
\tikzstyle{input} = [coordinate]
\tikzstyle{output} = [coordinate]


\begin{document}

\bibliographystyle{IEEEtran}
\vspace{3cm}

\title{5.13.74}
\author{EE25BTECH11049 - Sai Krishna Bakki}
\maketitle
\vspace{-3em}
% \newpage
% \bigskip
{\let\newpage\relax\maketitle}

\renewcommand{\thefigure}{\theenumi}
\renewcommand{\thetable}{\theenumi}
\setlength{\intextsep}{10pt} % Space between text and floats


\numberwithin{equation}{enumi}
\numberwithin{figure}{enumi}
\renewcommand{\thetable}{\theenumi}

\textbf{Question:}\\    
The trace of a square matrix is defined to be the sum of its diagonal entries. If $\vec{A}$ is a 2 x 2 matrix, such that the trace of $\vec{A}$ is 3 and the trace of $\vec{A}^{3}$ is -18, then the value of the determinant of $\vec{A}$ is\\
\textbf{Solution:}\\
Given:
\begin{align}
    \operatorname{tr}(\vec{A})&=3\\
    \operatorname{tr}(\vec{A^3})&=-18\\
    \operatorname{tr}(\vec{I})&=2
\end{align}
Let the eigenvalues of 2x2 matrix A be $\lambda_1$ and $\lambda_2$, we know that trace is the sum of eigenvalues.
\begin{align}
\lambda_1 +\lambda_2 &=3
\label{eq1}
\end{align} 
we are given that $\operatorname{tr}(\vec{A^3})&=-18$. Since the eigenvalues of $\vec{A}^3$ are $\lambda_1^3$ and $\lambda_2^3$, the trace of $\vec{A}^3$ is their sum.
\begin{align}
    \lambda_1^3 + \lambda_2^3 &=-18
    \label{eq2}
\end{align}
We can use the algebraic identity for the sum of cubes to connect our two equations $\eqref{eq1}$ and $\eqref{eq2}$.
\begin{align}
    \lambda_1^3 + \lambda_2^3 &=(\lambda_1+\lambda_2)((\lambda_1+\lambda_2)^2-3\lambda_1\lambda_2)
\end{align}
Substituting the equations $\eqref{eq1}$ and $\eqref{eq2}$ in above equation, we get
\begin{align}
\lambda_1\lambda_2&=5
\end{align}
But determinant of $\vec{A}$ is $\lambda_1\lambda_2$.\\
Therefore, the value of the determinant of $\vec{A}$ is 5.
\end{document}
