\documentclass{beamer}
\usepackage[utf8]{inputenc}

\usetheme{Madrid}
\usecolortheme{default}
\usepackage{amsmath,amssymb,amsfonts,amsthm}
\usepackage{txfonts}
\usepackage{tkz-euclide}
\usepackage{listings}
\usepackage{adjustbox}
\usepackage{array}
\usepackage{tabularx}
\usepackage{gvv}
\usepackage{lmodern}
\usepackage{circuitikz}
\usepackage{tikz}
\usepackage{graphicx}
\usepackage{textcomp}
\usepackage{cancel}
\setbeamertemplate{page number in head/foot}[totalframenumber]

\usepackage{tcolorbox}
\tcbuselibrary{minted,breakable,xparse,skins}



\definecolor{bg}{gray}{0.95}
\DeclareTCBListing{mintedbox}{O{}m!O{}}{%
  breakable=true,
  listing engine=minted,
  listing only,
  minted language=#2,
  minted style=default,
  minted options={%
    linenos,
    gobble=0,
    breaklines=true,
    breakafter=,,
    fontsize=\small,
    numbersep=8pt,
    #1},
  boxsep=0pt,
  left skip=0pt,
  right skip=0pt,
  left=25pt,
  right=0pt,
  top=3pt,
  bottom=3pt,
  arc=5pt,
  leftrule=0pt,
  rightrule=0pt,
  bottomrule=2pt,
  toprule=2pt,
  colback=bg,
  colframe=orange!70,
  enhanced,
  overlay={%
    \begin{tcbclipinterior}
    \fill[orange!20!white] (frame.south west) rectangle ([xshift=20pt]frame.north west);
    \end{tcbclipinterior}},
  #3,
}
\lstset{
    language=C,
    basicstyle=\ttfamily\small,
    keywordstyle=\color{blue},
    stringstyle=\color{orange},
    commentstyle=\color{green!60!black},
    numbers=left,
    numberstyle=\tiny\color{gray},
    breaklines=true,
    showstringspaces=false,
}
%------------------------------------------------------------
%This block of code defines the information to appear in the
%Title page
\title %optional
{5.13.74}
\date{}
%\subtitle{A short story}

\author % (optional)
{Sai Krishna Bakki - EE25BTECH11049}

\begin{document}
\frame{\titlepage}
\begin{frame}{Question}
The trace of a square matrix is defined to be the sum of its diagonal entries. If $\vec{A}$ is a 2 x 2 matrix, such that the trace of $\vec{A}$ is 3 and the trace of $\vec{A}^{3}$ is -18, then the value of the determinant of $\vec{A}$ is
\end{frame}
\begin{frame}{Theoretical Solution}
    Given:
\begin{align}
    \operatorname{tr}(\vec{A})&=3\\
    \operatorname{tr}(\vec{A^3})&=-18\\
    \operatorname{tr}(\vec{I})&=2
\end{align}
Using Cayley-Hamilton Theorem, we know that
\begin{align}
    \mydet{\vec{A}-\lambda\vec{I}}&=0,
    \text{for $\lambda=\vec{A}$}
\end{align}
For a 2 x 2 matrix $\vec{A}$, the characteristic equation is:
\begin{align}
    \lambda^{2}-\operatorname{tr}(\vec{A})\lambda+\mydet{\vec{A}}&=0
\end{align}
According to the theorem, the matrix $\vec{A}$ itself will satisfy this equation:
\begin{align}
    \vec{A^2}-\operatorname{tr}(\vec{A})\vec{A}+\mydet{\vec{A}}\vec{I}&=0\\
        \vec{A^2}-3\vec{A}+\mydet{\vec{A}}\vec{I}&=0\\
        \vec{A^2}&=3\vec{A}-\mydet{\vec{A}}\vec{I}
\end{align}
\end{frame}
\begin{frame}{Theoretical Solution}
To find $\vec{A^3}$, we multiply the equation by $\vec{A}$
\begin{align}
    % \vec{A^3}=\vec{A}\brak{3\vec{A}-\mydet{\vec{A}}\vec{I}}\\
    \vec{A^3}=3\vec{A^2}-\mydet{\vec{A}}\vec{A}
\end{align}
Now, substitute the expression for $\vec{A^2}$ into the equation for $\vec{A^3}$:
\begin{align}
\vec{A^3}=3\brak{3\vec{A}-\mydet{\vec{A}}\vec{I}}-\mydet{\vec{A}}\vec{A}\\
\vec{A^3}=\brak{9-\mydet{\vec{A}}}\vec{A}-3\mydet{\vec{A}}\vec{I}
\end{align}
Let's take the trace of both sides of this equation. Using the linearity properties of the trace 
\begin{align}
    \operatorname{tr}(\vec{X}+\vec{Y})=\operatorname{tr}(\vec{X})+\operatorname{tr}(\vec{Y})\\
    \operatorname{tr}(k\vec{X})=k\operatorname{tr}(\vec{X})\\
    \operatorname{tr}(\vec{A^3})=\operatorname{tr}\brak{\brak{9-\mydet{\vec{A}}}\vec{A}-3\mydet{\vec{A}}\vec{I}}\\
    \operatorname{tr}(\vec{A^3})=\brak{9-\mydet{\vec{A}}}\operatorname{tr}(\vec{A})-3\mydet{\vec{A}}\operatorname{tr}(\vec{I})
\end{align}
\end{frame}
\begin{frame}{Theoretical Solution}
Substituting equations (0.1) and (0.2) in above equation, we get
\begin{align}
    -18=\brak{9-\mydet{\vec{A}}}(3)-3\mydet{\vec{A}}(2)\\
    \therefore \mydet{\vec{A}}=5
\end{align}
\end{frame}
\begin{frame}[fragile]
\frametitle{C Code}
\begin{lstlisting}
    #include <math.h>

/*
    This function calculates the determinant of a 2x2 matrix given the
    trace of the matrix (trace_A) and the trace of its cube (trace_A3).

    The formula is derived from the relationship between the trace,
    determinant, and eigenvalues of a matrix.

    For Windows DLL compilation, __declspec(dllexport) is used to
    export the function, making it visible to other programs. For
    Linux/macOS, this is not strictly necessary when compiling with -fPIC.
*/
#if defined(_WIN32)
    #define DLLEXPORT __declspec(dllexport)
    \end{lstlisting}
    \end{frame}
    \begin{frame}[fragile]
    \frametitle{C Code}
    \begin{lstlisting}
#else
    #define DLLEXPORT
#endif

DLLEXPORT double solve_determinant_2x2(double trace_A, double trace_A3) {
    // Ensure we don't divide by zero if trace_A is 0.
    if (trace_A == 0) {
        return 0.0; // Or handle as an error, e.g., return NAN.
    }
    
    // Formula: det(A) = (tr(A)^3 - tr(A^3)) / (3 * tr(A))
    return (pow(trace_A, 3) - trace_A3) / (3.0 * trace_A);
}
    \end{lstlisting}
    \end{frame}
    \begin{frame}[fragile]
    \frametitle{Python Code Through Shared Output}
    \begin{lstlisting}
    import ctypes
import os

# Define the name of the shared library based on the operating system
if os.name == 'nt': # Windows
    lib_name = 'solver.dll'
else: # Linux, macOS, etc.
    lib_name = 'solver.so'

# Construct the full path to the library file in the current directory
lib_path = os.path.join(os.path.dirname(os.path.abspath(__file__)), lib_name)

try:
    # 1. Load the shared library
    solver_lib = ctypes.CDLL(lib_path)
except OSError as e:
    \end{lstlisting}
    \end{frame}
    \begin{frame}[fragile]
    \frametitle{Python Code Through Shared Output}
    \begin{lstlisting}
    print(f"Error: Could not load the shared library '{lib_name}'.")
    print("Please make sure you have compiled the C code first.")
    print(f"Details: {e}")
    exit()

# 2. Define the function signature to match the C code
#    Specify the argument types (argtypes)
solver_lib.solve_determinant_2x2.argtypes = [ctypes.c_double, ctypes.c_double]
#    Specify the return type (restype)
solver_lib.solve_determinant_2x2.restype = ctypes.c_double

# 3. Define the input values from the problem
trace_A = 3.0
trace_A3 = -18.0

# 4. Call the C function from Python
    \end{lstlisting}
    \end{frame}
    \begin{frame}[fragile]
    \frametitle{Python Code Through Shared Output}
    \begin{lstlisting}
#    Python floats will be automatically converted to ctypes.c_double
determinant = solver_lib.solve_determinant_2x2(trace_A, trace_A3)

# 5. Print the result
print("--- Calling C function from Python using ctypes ---")
print(f"Given tr(A) = {trace_A}")
print(f"Given tr(A^3) = {trace_A3}")
print("-" * 25)
print(f"The calculated determinant of A is: {determinant}")
    \end{lstlisting}
    \end{frame}
    \begin{frame}[fragile]
    \frametitle{Python Code}
    \begin{lstlisting}
import sympy

# 1. Define the unknown variable and knowns as symbolic objects
# d represents the determinant of A, which we want to find.
d = sympy.Symbol('d')

# tr_A is the trace of A.
tr_A = 3
# tr_A3 is the trace of A^3.
tr_A3 = -18
# For a 2x2 matrix, the trace of the identity matrix (I) is 2.
tr_I = 2

# 2. Set up the equation based on the Cayley-Hamilton theorem
# The derived formula is: tr(A^3) = (tr(A)**2 - d)*tr(A) - d*tr(A)*tr(I)
    \end{lstlisting}
    \end{frame}
    \begin{frame}[fragile]
    \frametitle{Python Code}
    \begin{lstlisting}
# We create an equation object that is equal to zero.
equation = sympy.Eq((tr_A**2 - d)*tr_A - d*tr_A*tr_I, tr_A3)

# 3. Solve the equation for our unknown variable 'd'
# sympy.solve takes the equation and the variable to solve for.
solution = sympy.solve(equation, d)

# 4. Print the result
# The solution is a list, so we print the first element.
print(f"The equation to solve is: {equation}")
print(f"The calculated determinant of A is: {solution[0]}")
\end{lstlisting}
\end{frame}
\end{document}